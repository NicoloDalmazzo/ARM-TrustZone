\chapter{Arm TrustZone}
\label{sec:trustzone}

TrustZone is the first attempt to implement a confidential computing support within the architecture of the Arm A and M CPU profiles \cite{arm-102418}.
This support enables two security states:
\begin{itemize}
\item \textbf{Non-secure state}, where it runs the Rich OS such as Linux, it is called the Rich Execution Environment (REE).
\item \textbf{Secure state}, where the Trusted Execution Environment (TEE) is executed and contains a limited software stack in order to reduce the attack surface.
\end{itemize}

The current document will be focused only on the Arm TrustZone technology available on the A profile CPUs. Note that the implementation for the M profile CPUs present several differences.

\section{Secure World}
\subsection{Secure Monitor}
In order to access the Secure state, the requests have to be approved and checked by a trusted monitor that handles every request to the Secure state (Secure Monitor). Handling the change between the security states, the monitor has to store the context in which the Processing Element it was before changing the state, so the context can be restored and the normal execution can resume where it was stopped.

Secure Monitor is found at the highest exception level, so every time a state change is required, an exception that has to be handled at the highest level is required. These types of request are called Secure Monitor Calls (SMC) and can be raised by the Rich OS or the Hypervisor itself.

\subsection{Memory Isolation}
To guarantee isolation between the Secure state and the Non-secure state, memory isolation has been developed in order to disable the access from the Normal world to the secure memory, by means of different translation regime and define the owner state of each memory location. The memory isolation blocks the access of secure memory by applications that are running in the non-secure state, however a service that is in secure state can access both (secure and non-secure) memories. 

\section{Attestation in TrustZone}
The Arm TrustZone technology does not provide a standard for the attestation process. So, there are many different solution, highly implementation specific, that address the issue in different ways.
The work of Ménétrey et al. \cite{MnTrey2022} presents, among several confidential computing technology, a description of the state of art of the attestation process for Arm TrustZone. From the research, it emerges that there are several different implementations. Moreover, some attestation proposal enforce the mutual attestation of both the end-points of a communication channel. These efforts are sounds since Arm TrustZone is mostly adopted, besides the smartphone market, in the IoT sector. For instance, mutual attestation could be a great solution for providing device authentication in a network of sensors that send their data to a master node. However, the lack of an official standard, along with the strong economic trade-off of the IoT devices, could lead to not strong secure implementations of the attestation mechanism.

\section{Downsides}

Arm TrustZone suffers from several problems that are architecture based and so they cannot be fixed by architecture extensions:
\begin{itemize}
    \item a single TEE is present, where more than one application could run. So, there is no mutual distrust between Trusted applications by hardware enforcement.
    \item Secure Monitor shares the same Physical Address Space of the Secure state, so if an application that runs in it is compromised, the overall system could be attacked.
\end{itemize}