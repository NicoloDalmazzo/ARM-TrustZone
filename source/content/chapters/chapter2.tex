\chapter{Arm architectures details}
\label{sec:arch}

\section{Exception Model}
In the AArch64 architecture there are present different level of privilege. The current privilege level can change only when the processor takes an exception or returns from an exception. For this behavior, in the Arm architecture, the privilege model is referred as the Exception Model \cite{arm-102412}.

\subsection{Exception Levels}
\label{subsec:exc_levels}
The AArch64 provides four exception levels and they are normally abbreviated to as EL<x>, where x indicates the number of the privilege level. The four privilege levels are presented in the following, from the least privilege to the highest:
\begin{itemize}
    \item \textbf{Exception Level 0 (EL0)}, where user applications run;
    \item \textbf{Exception Level 1 (EL1)}, where the operating system runs;
    \item \textbf{Exception Level 2 (EL2)}, if present the hypervisor is running here;
    \item \textbf{Exception Level 3 (EL3)}, the highest privilege level where only secure firmware can run.
\end{itemize}

In order to change the current privilege level, the processor has to take an exception, resulting in the same previous privilege level or an higher one, never a lower privilege level. Instead, when returning from an exception, the privilege level can stay the same, or go to a lower one. It is the inverse process when the exception was taken.

\subsection{Types of Privilege}
In the AArch64 architecture there are two relevant types of privilege:
\begin{itemize}
    \item privilege in the memory system;
    \item privilege regarding access to processor resources.
\end{itemize}

The A-profile of the Arm architecture allows, thanks to the MMU, defining attributes to region of memory via software. These attributes enable to define separate access permissions for privileged and unprivileged accesses.

The System registers store the combination of settings that define the current processor context. For each type of register a privilege level is required to access and manage it. In the Arm architecture to define which is the minimum level of privilege required, the System registers are named with the Exception level suffix, such as for example the System Control Register (SCTLR) is defined for each exception level and so there are present SCTLR\_EL1, SCTLR\_EL2 and SCTLR\_EL3 registers. Note that if the register name differs only from the suffix, these registers are implemented separately in hardware. The suffix defines the minimum privilege level, so a register of EL1 can be accessed by the same EL1 or higher levels (EL2 or EL3).

\section{Memory Management}
In AArch64 there are several independent Virtual Address Spaces. Each space is referred to an exception level (section \ref{subsec:exc_levels}) and a security state (section \ref{subsec:sec_states}). Each virtual address space is independent and it has its own settings and tables, referred to as "translation regimes" \cite{arm-101811}.

In order to identify to which space the address is referred to, a prefix is needed. For example, NS.EL2:0x8000 refers to the address 0x8000 in the Non-secure EL2 virtual address space.

\subsection{Memory Management Unit}
The Memory Management Unit (MMU) is responsible for the translation of virtual addresses used by software to physical addresses used in the memory system.

The core process of translation is based on the translation tables, where each entry correspond to a equal-sized block of virtual address space and it contains the address of a physical memory block and the attributes to use when accessing the physical address.

As it will be seen in section \ref{subsec:mem_isolation}, the MMU will perform checks on the current security state of the processor to determine if the translated physical address could be accessed or not, enabling memory isolation.

\section{Granule Partition Table}
A translation granule is the smallest block of memory that can be described. Smaller blocks cannot be described, only larger ones, multiples of the granule. The AArch64 supports different granule size: 4KB, 16KB and 64KB \cite{arm-101811}.

A set of Granule Partition Tables (GPT) associates to each granule the corresponding Physical Address Space (PAS), enabling the dynamic allocation of memory regions between different PASs. See in details in section \ref{subsubsec:granule}.